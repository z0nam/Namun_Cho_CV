%%%%%%%%%%%%%%%%%%%%%%%%%%%%%%%%%%%%%%%%%
% Compact Academic CV
% LaTeX Template
% Version 1.0 (10/6/2012)
%
% This template has been downloaded from:
% http://www.LaTeXTemplates.com
%
% Original author:
% Dario Taraborelli (http://nitens.org/taraborelli/home)
%
% License:
% CC BY-NC-SA 3.0 (http://creativecommons.org/licenses/by-nc-sa/3.0/)
%
% Important:
% This template needs to be compiled using XeLaTeX
%
% Note: this template has the option to use the Hoefler Text font, see the
% font configurations section below for instructions on using this font
%
%%%%%%%%%%%%%%%%%%%%%%%%%%%%%%%%%%%%%%%%%

%----------------------------------------------------------------------------------------
%	PACKAGES AND OTHER DOCUMENT CONFIGURATIONS
%----------------------------------------------------------------------------------------

\documentclass[11pt, a4paper]{article} % Document font size and paper size
\usepackage[hangul]{kotex}
\usepackage{fontspec} % Allows the use of OpenType fonts


\usepackage{geometry} % Allows the configuration of document margins
\geometry{a4paper, textwidth=5.5in, textheight=8.5in, marginparsep=7pt, marginparwidth=.6in} % Document margin settings
\setlength\parindent{0in} % Remove paragraph indentation

\usepackage[usenames,dvipsnames]{xcolor} % Custom colors

\usepackage{sectsty} % Allows changing the font options for sections in a document
\usepackage[normalem]{ulem} % Custom underlining
\usepackage{xunicode} % Allows generation of unicode characters from accented glyphs
\defaultfontfeatures{Mapping=tex-text} % Converts LaTeX specials (``quotes'' --- dashes etc.) to unicode

\usepackage{marginnote} % For margin years
\newcommand{\years}[1]{\marginnote{\scriptsize #1}} % New command for including margin years
\renewcommand*{\raggedleftmarginnote}{}
\setlength{\marginparsep}{7pt} % Slightly increase the distance of the margin years from the contant
\reversemarginpar

\usepackage[xetex, bookmarks, colorlinks, breaklinks, pdftitle={조남운 - vita},pdfauthor={조남운}]{hyperref} % PDF setup - set your name and the title of the document to be incorporated into the final PDF file meta-information
\hypersetup{linkcolor=blue,citecolor=blue,filecolor=black,urlcolor=MidnightBlue} % Link colors

%----------------------------------------------------------------------------------------
%	FONT CONFIGURATIONS
%----------------------------------------------------------------------------------------

% Two font choices are available in this template, the default is Linux Libertine, available for free at: http://www.linuxlibertine.org while the secondary choice is Hoefler Text which comes bundled with Mac OSX.
% To use Hoefler Text, comment out the Linux Libertine block below and uncomment the Hoefler Text block. You will also need to replace the "\&" characters with "\amper{}" in section titles.

% Linux Libertine Font (default)
% \setromanfont [Ligatures={Common}, Numbers={OldStyle}, Variant=01]{Linux Libertine O} % Main text font
%\setmonofont[Scale=0.8]{Monaco} % Set mono font (e.g. phone numbers)
% \sectionfont{\mdseries\upshape\Large} % Set font options for sections
% \subsectionfont{\mdseries\scshape\normalsize} % Set font options for subsections
% \subsubsectionfont{\mdseries\upshape\large} % Set font options for subsubsections
% \chardef\&="E050 % Custom ampersand character

% Hoefler Text Font (bundled with Mac OSX)
\setromanfont [Ligatures={Common}, Numbers={OldStyle}]{Hoefler Text} % Main text font
\setmonofont[Scale=0.8]{Monaco} % Set mono font (e.g. phone numbers)
\setsansfont[Scale=0.9]{Optima Regular} % Set sans font, used in the main name and titles in the document
\newcommand{\amper}{{\fontspec[Scale=.95]{Hoefler Text}\selectfont\itshape\&}} % Custom ampersand character
\sectionfont{\sffamily\mdseries\large\underline} % Set font options for sections
\subsectionfont{\rmfamily\mdseries\scshape\normalsize} % Set font options for subsections
\subsubsectionfont{\rmfamily\bfseries\upshape\normalsize} % Set font options for subsubsections
\setmainhangulfont[ItalicFont={*},ItalicFeatures={FakeSlant=.167}]{NanumMyeongjo}

%----------------------------------------------------------------------------------------

\begin{document}

%----------------------------------------------------------------------------------------
%	CONTACT AND GENERAL INFORMATION SECTION
%----------------------------------------------------------------------------------------

{\LARGE 조남운}\\[1cm] % Your name
한국행동경제학연구소\\ % Your address
KBERI (Korea Behavioral Economics Research Institute)\\
서울특별시 서초구 반포대로30길 81 웅진타워 1102호. \texttt{우:06644}\\
\\[.2cm]
Phone: \texttt{+82-10-6343-2884}\\ % Your phone number
% Fax: \texttt{+82-2-888-4454}\\[.2cm] % Your fax number
Fax: \texttt{+82-2-539-5412}\\[.2cm] % Your fax number
Email: \href{mailto:namun.cho@gmail.com}{namun.cho@gmail.com}\\ % Your email address
\textsc{url}: \href{https://github.com/z0nam}{https://github.com/z0nam}\\ % Your academic/personal website

% {\LARGE 조남운}\\[1cm] % Your name
% 서울대학교 분배정의와 사회통합 연구센터\\ % Your address
% 서울시 관악구 관악로 1 서울대학교 16동 M304-2호 \texttt{08826}
% \\[.2cm]
% Phone: \texttt{010-6343-2884}\\ % Your phone number
% % Fax: \texttt{02-880-6454}\\[.2cm] % Your fax number
% Email: \href{mailto:namun.cho@gmail.com}{namun.cho@gmail.com}\\ % Your email address
% \textsc{url}: \href{http://spsm.snu.ac.kr/namun/}{http://spsm.snu.ac.kr/namun/}\\ % Your academic/personal website

% \vfill % Whitespace between contact information and specific CV information

%------------------------------------------------

% Born: March 12, 1879---Ulm, Germany\\ % Your date of birth
% Nationality: German/American % Your nationality

%------------------------------------------------

\section*{현재 직위}

% \emph{박사후 연구원}, 분배정의와 사회통합 SSK 연구센터 % Your current or previous employment position
% \emph{강사}, 고려대학교 경제학과
\emph{수석연구원}, 한국행동경제학연구소

%------------------------------------------------

\section*{관심분야}

Computational Economics; Experimental Economics; Evolutionary Game Theory; Agent-based Model. % Your primary areas of research interest

%----------------------------------------------------------------------------------------
%	WORK EXPERIENCE SECTION
%----------------------------------------------------------------------------------------


%----------------------------------------------------------------------------------------
%	EDUCATION SECTION
%----------------------------------------------------------------------------------------

\section*{학력 및 경력}

\years{2001} \textsc{(학사)} 서울대학교 컴퓨터공학과\\
\years{2006} \textsc{(석사)} 고려대학교 경제학과\\
\years{2012} \textsc{(박사)} 고려대학교 경제학과\\
\years{2012} \textsc{(위촉 연구원)} \href{http://isdpr.org}{서울대학교 사회발전연구소 Social Network Computing Center}\\
\years{2013} \textsc{(박사후 연구원)} \href{http://cdj.snu.ac.kr}{분배정의와 사회통합 SSK 연구센터}\\
\years{2016} \textsc{(연구원)} \href{http://kor.kias.re.kr}{한국고등과학원 초학제연구 프로그램 (게임이론)}\\
\years{2017} \textsc{(연수연구원)} \href{http://devsocial.snu.kr/ier}{서울대학교 경제연구소}\\
\years{2018} \textsc{(수석연구원)} \href{http://kberi.wordpress.com}{한국행동경제학연구소}
\newpage
%----------------------------------------------------------------------------------------
%	GRANTS, HONORS AND AWARDS SECTION
%----------------------------------------------------------------------------------------

\section*{수상경력}

\years{2010} 제5회 복잡계 컨퍼런스 대학원생부문 최우수논문\\
\years{2011} 제6회 복잡계 컨퍼런스 대학원생부문 동상

%----------------------------------------------------------------------------------------
%	PUBLICATIONS AND TALKS SECTION
%----------------------------------------------------------------------------------------

\section*{주요 연구실적}

\years{2012} \href{http://spsm.snu.ac.kr/namun/papers/phDdraft.full.pdf}{``Three Essays in the Computational and Experimental Economics''}, \emph{PhD Dissertation}\\

\subsection*{Journal article}


\years{2019} \href{https://link.springer.com/article/10.1007/s10614-017-9705-5}{Asset Market Volatility and New Keynesian Macroeconomics: A Game-Theoretic Approach (자산시장 변동성과 새 케인지언 거시경제학: 게임이론적 접근)}. \emph{Computational Economcis}, 54:245. (장태석과 함께 씀 (제1저자))\\
\years{2018}\href{https://www.kci.go.kr/kciportal/ci/sereArticleSearch/ciSereArtiView.kci?sereArticleSearchBean.artiId=ART002375699}{경제 실험의 환경 및 보상이 참여 동기에 미치는 영향}. \emph{경제교육연구}, 25(2), 127--156. (신우진과 함께 씀 (제1저자))\\
\years{2016} \href{https://www.kci.go.kr/kciportal/ci/sereArticleSearch/ciSereArtiView.kci?sereArticleSearchBean.artiId=ART002118543}{행위자 기반 시뮬레이션을 통한 공공 연구개발비 지출 제도 비교 연구}. \emph{산업혁신연구}, 32(2), 227–-255. (이남형, 강영준, 최은철, 김아미와 함께 씀(교신저자))\\
\years{2016}\href{https://www.kci.go.kr/kciportal/ci/sereArticleSearch/ciSereArtiView.kci?sereArticleSearchBean.artiId=ART002164687}{선거예측시장에서의 당파적 거래: 2012 대선 주식시장에 대한 보고}. \emph{한국정치연구}, 25(3), 197--223. (박원호, 한규섭, 안도경과 함께 씀 (제1저자))
%


%------------------------------------------------


\subsection*{Book Chapter}

% (역서)소셜리스트 레지스터
% isbn 붙은 연구들 - 행동경제학의 활용방안, 고소제도연구, 대안교육
\years{2014} \href{http://www.springer.com/us/book/9781461461692}{``NetMiner'' (넷마이너)}, in Alhajj, Reda, Rokne, Jon (Eds.), \textit{Encyclopedia of Social Network Analysis and Mining (사회연결망분석과 데이터마이닝 백과사전)}, Springer-Verlag, NY, ISBN 978-1-4614-6169-2 (with Ghi-Hoon Ghim, Jeonsu Seo)\\

%------------------------------------------------

\subsection*{Magazine Article}

\years{2017}
\href{http://futures.kaist.ac.kr/center05/periodical/view/tableid/periodical/id/384}{``미국은 대선을 어떻게 예측하는가''}, \emph{See Futures} 2017년 봄호\\
\years{2015} \href{http://webzine.kps.or.kr/?now_url=../contents/data/webzine/webzine/14762088365.pdf}{``정치예측시장의 작동원리''}, \emph{물리학과 첨단기술} 2015년 1/2월호 (안도경과 함께 씀)\\

\subsection*{Conference Participations}
\years{2020}
\href{https://www.dropbox.com/s/frw2wreosoyt7zd/coopedu_ksesa.pdf?dl=0}{``신뢰게임으로 측정한 협동형 대안교육의 사회경제적 가치''}, 한국사회경제학회 2020년 가을 학술대회\\
\years{2018}
\href{http://ames.sogang.ac.kr/dwonload/pdf/paper_399.pdf}{``Dynamics in Repeated Game with
Cumulative Payoff and Portfolio Strategy:
Hawk-Dove and Public Goods Game'' (누적적 보상과 포트폴리오 전략 하에서의 반복게임 동학: 매-비둘기 게임과 공공재게임을 중심으로)}, AMES 2018\\
\years{2018}
\href{https://editorialexpress.com/cgi-bin/conference/download.cgi?db_name=SCW2018&paper_id=269}{``Dynamics in Repeated Game with
Cumulative Payoff and Portfolio Strategy:
Hawk-Dove and Public Goods Game'' (누적적 보상과 포트폴리오 전략 하에서의 반복게임 동학: 매-비둘기 게임과 공공재게임을 중심으로)}, SSCW 2018\\
\years{2016} \href{http://spsm.snu.ac.kr/namun/papers/abm.pdf}{``경제학에서의 행위자기반모형: 소개와 전망''}, 2016년 경제학 공동 학술대회 (한국경제학사학회 세션)\\
\years{2014} \href{https://www.dropbox.com/s/5i99u86w7dkxgsb/tobacco_2014_KIPF_draft.pdf?dl=0}{``담배상품의 특수성을 감안한 소비모형 검토''}, 2014년 재정패널 학술대회\\
\years{2014} \href{https://www.dropbox.com/s/utbilo9ib0cwgiq/inv_2014s.pdf?dl=0}{``느슨한 웹 기반 행동실험에서의 금전적 보상과 비금전적 보상에 관하여''}, 2014년 계량경제학회 하계학술대회\\
\years{2014} \href{https://www.dropbox.com/s/gshcjhu31guh5mv/mpsa_false_consensus_media_effect.pdf?dl=0}{``A Media-based Explanation of the False Consensus Effect: Evidence from a Presidential Election Prediction Market'' (허구적 일치성 효과에 대한 미디어 기반 해석: 대선 예측시장으로부터의 근거)}, Midwest Political Science Association (MPSA) 2014. (with Kyu S. Hahn, T.K. Ahn, Won-Ho Park and Hyeri Shin)\\
\years{2014} \href{https://www.dropbox.com/s/qaq2p08mq5sn4tx/mpsa_partisan_trading_spsm_ver3.0.docx?dl=0}{``Partisan Trading in a Presidential Election Prediction Market Experiment: The Case of 2012 Korean Presidential Election'' (대선 예측시장에서 나타난 당파적 거래: 2012년 한국 대통령 선거 사례)}, MPSA 2014. (with Won-ho Park, TK Ahn, and Kyu S. Hahn)\\
\years{2013} \href{https://www.dropbox.com/s/g37bsl1g2smuhb3/%5B%EB%8B%B4%EB%B0%B0%EA%B3%B5%EB%8F%99%EC%97%B0%EA%B5%AC%20%ED%95%99%ED%9A%8C%EB%B0%9C%ED%91%9C%EC%9B%90%EA%B3%A0%20%EC%B5%9C%EC%A2%85%EC%88%98%EC%A0%95%EB%B3%B8%5Dcho_go_kang%282013_10_10%29.pdf?dl=0}{``담배가격 인상의 소득귀착효과 연구''}, 2013 재정학회 추계 학술대회. (강영준, 고제이와 함께 씀)\\
\years{2011} \href{https://www.dropbox.com/s/u57gbzq66ourzb2/Draft120611_zn.pdf?dl=0}{``최적 고속도로 요금 검토: 천안-논산과 대구-부산 구간을 중심으로''}, 2011년 경제학 공동 학술대회. (이남형, 김균과 함께 씀)\\
\years{2011} \href{https://www.dropbox.com/s/a9a4asadjeo36x7/NamunCho_COREN2011_draft%2820111114%29.pdf?dl=0}{``상호의존적 기대하에서 정보의 역할: 행위자 기반 시뮬레이션''}, 제6회 복잡계 컨퍼런스\\
\years{2010} \href{https://www.dropbox.com/s/jrceax37h2sxz4j/coren_draft_short.pdf?dl=0}{``상호의존적 기대하에서 정보의 역할: 실험적 접근''}, 제5회 복잡계 컨퍼런스\\
\years{2010} \href{https://www.dropbox.com/s/qejr4lf2klroa7d/draft%2811.04.19%29.pdf?dl=0}{``What makes the open source software development sustainable?: Agent based model on two conjectures'' (무엇이 오픈 소스 소프트웨어 개발을 지속가능케 하는가? 두 가지 맥락에 대한 행위자기반모형)}, The 4th FLOSS International Workshop on Free/Libre/Open Source Software. (with Namhyung Lee)\\

\subsection*{연구 프로젝트} % (fold)
\label{sec:연구보고서}

\subsubsection*{진행중인 연구용역} % (fold)
\label{ssub:진행중}
\years{2020} {``예방접종 전산등록 자료를 활용한 지역별 예방접종 현황 분석 및 예방접종 활성화 방안''}, 질병관리청, 30백만원, 공동연구자\\
\years{2020} {``신SW기술 대국민 인식개선을 위한 정책효과 연구''}, 소프트웨어정책연구소, 30백만원, 공동연구자\\
\years{2020} {``행동경제학 관점에서 본 포스트 코로나 시대의 서울시의회의 역할과 과제''}, 서울시의회, 29백만원, 연구책임자(PM)\\
% subsubsection 진행중 (end)

\subsubsection*{완료한 연구용역} % (fold)
\label{ssub:완료}
\years{2020} {``행동강화물품이 금연동기 강화 및 금연유지에 미치는 효과 분석''}, 보건복지부, 연구비 55.7백만원, 연구책임자 (PM)\\
\years{2020}\href{https://www.khealth.or.kr/board/view?pageNum=1&rowCnt=10&no1=508&linkId=999939&menuId=MENU00932&schType=0&schText=&boardStyle=&categoryId=&continent=&country=}{``흡연이 노동력 상실에 미치는 영향과 정책방향 연구''}, 한국건강증진개발원 국가금연지원센터, 연구비 83.5백만원, 연구책임자(PM)\\
\years{2020}\href{http://poll-mbc.co.kr}{``2020 총선 여론조사를 조사하다''}, MBC 보도부, 개발비 4백만원, 개발책임자(PM)\\
\years{2019}{``30세 이상 중장년층 헌혈 활성화를 위한 연구''}, 대한적십자사 혈액관리본부, 연구비 45.1백만원, 연구책임자(PM)\\
\years{2019}\href{https://opengov.seoul.go.kr/research/19072877}{``부모협동형 대안교육기관의 현황과 제도적 개선방안 연구''}, 서울시의회 교육위원회, 연구비 21.4백만원, 연구책임자(PM)\\
\years{2019}\href{https://women.na.go.kr:444/women/reference/reference01.do?mode=view&articleNo=660628&article.offset=0&articleLimit=10}{``행동경제학 관점에서 본 가부장제 인식 개선 방안''}, 국회 여성가족위원회, 연구비 10백만원, 연구책임자(PM)\\
\years{2018}``행동경제학 실험을 통한 서울시 대시민 경제 및 복지 정책 제도 개선 또는 도출 연구'', 서울연구원, 연구비 14.3백만원, 연구책임자(PM)\\
\years{2018}``행동경제학 실험을 통한 서울시 환경 및 에너지 정책 제도 개선 또는 도출 연구'', 서울연구원, 연구비 14.3백만원, 연구책임자(PM)\\
\years{2017}\href{https://www.dbpia.co.kr/journal/articleDetail?nodeId=NODE07577597}{``고소제도의 운영현황과 개선방안에 관한 연구''}, 한국형사정책연구원, 연구비 6백만원, 공동연구원\\

% subsubsection 완료 (end)

% section 연구보고서 (end)

%----------------------------------------------------------------------------------------
%	TEACHING SECTION
%----------------------------------------------------------------------------------------

\section*{Teaching}

\years{2007-현재} 경제수학, 계량경제학, 게임이론, 미시경제학, 거시경제학, 경제원론, 경제학개론, 국제경제학, 토지정책론 강의

%------------------------------------------------

\section*{Programming Skills}

JAVA, Stata, PHP, Javascript, Python, Ruby, R, NetLogo, oTree (Python django), zTree

github profile: \url{https://github.com/z0nam}


% \section*{Service to the profession}
%
% ...



\vfill{} % Whitespace before final footer

%----------------------------------------------------------------------------------------
%	FINAL FOOTER
%----------------------------------------------------------------------------------------

\begin{center}
{\scriptsize Last updated: \today%\- •\- \href{http://www.LaTeXTemplates.com}{http://www.LaTeXTemplates.com}
} % Any final footer text such as a URL to the latest version of your CV, last updated date, compiled in XeTeX, etc
\end{center}

%----------------------------------------------------------------------------------------

\end{document}
